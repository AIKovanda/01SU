\chapter{úvod}

Umělá inteligence je systém, který umí vnímat vnější prostředí, vyhodnocovat ho, rozhodovat se a vykonat akci ke splnění cíle stanoveného uživatelem.

Vnímání většinou probíhá na základě obrazových dat pomocí sensorů. Vyhodnocování je pak ve stylu rozpoznání objektů, které jsou obsaženy v obrázcích apod.

Rozhodování je pak možno chápat jako optimalizační problém při okrajových podmínkách, např. u samořídících aut víme, že nelétají apod. 

Existuje \textbf{obecná} umělá inteligence, která prozatím patří mezi scifi a je to inteligence schopná všeho, co umí člověk nebo jiná inteligentní bytost. V dnešní době se pracuje se \textbf{speciální} umělá inteligence schopná řešit pouze konkrétní problémy.

Strojové učení se dále dělí na \textbf{klasické metody} (handcrafted features) a \textbf{hluboké učení} a jiných metod. 
